\documentclass[8pt]{article}
\usepackage{graphicx} % Required for inserting images
\usepackage{amsthm}
\usepackage{amsmath}
\usepackage{amsfonts}
\usepackage{amsopn}
\usepackage{comment}
\usepackage{amssymb}
\usepackage{hyperref}
\usepackage{bussproofs}
\usepackage[all, 2cell]{xy}
\usepackage[all]{xy}
\usepackage{rotating}
\usepackage{lscape}
\usepackage{minted}
\usepackage{dsfont}

\theoremstyle{definition}
\newtheorem{definition}{Definition}[section]

\theoremstyle{definition}
\newtheorem{theorem}{Theorem}[section]

\theoremstyle{definition}
\newtheorem{claim}{Claim}[section]

\theoremstyle{definition}
\newtheorem{ex}{Example}[section] 

\theoremstyle{definition}
\newtheorem{cons}{Construction}[section] 

\theoremstyle{definition}
\newtheorem{rem}{Remark}[section] 


\theoremstyle{definition}
\newtheorem{prop}{Proposition}[section]

\theoremstyle{definition}
\newtheorem{lemma}{Lemma}[section]

\theoremstyle{definition}
\newtheorem{fact}{Fact}[section]

\theoremstyle{definition}
\newtheorem{remark}{Remark}[section]

\theoremstyle{definition}
\newtheorem{notation}{Notation}[section]

\theoremstyle{definition}
\newtheorem{example}{Example}[section]

\theoremstyle{definition}
\newtheorem{col}{Corollary}[section]

\theoremstyle{question}
\newtheorem{question}{Question}

\let\strokeL\L
\renewcommand\L{\mathbf{L}}

\DeclareMathOperator*{\colim}{\operatorname{colim}}
\newcommand{\Ob}[1]{\operatorname{Ob}({\mathcal{#1}})}
\newcommand{\Cov}[1]{\operatorname{Cov}(#1)}


\title{Semantic and Proof-Theoretic Investigation of Noncommutative Geometric Logic}
\author{Daniel Rogozin}
\date{ }

\begin{document}

\maketitle

\section{Background}

\section{Definitions}

\section{Quantale, Cover Systems and Other Concepts}

\begin{definition}
  A \emph{quantale} is a structure $\mathcal{Q} = (Q, \bigvee, \cdot, \varepsilon)$
  where $(Q, \cdot, \varepsilon)$ is a monoid and $(Q, \bigvee)$ is a sup-lattice such that the following holds
  for each $a \in Q$ and for each indexed family $\{ a_i \in Q \: | \: i \in I \}$
  \begin{center}
    $a \cdot \bigvee \limits_{i \in I} a_i = \bigvee \limits_{i \in I} (a \cdot a_i)$

    $\bigvee \limits_{i \in I} a_i \cdot a = \bigvee \limits_{i \in I} (a_i \cdot a)$
  \end{center}
\end{definition}

\begin{definition}
Let $\mathcal{Q}$ be a quantale, a function $j : \mathcal{Q} \to \mathcal{Q}$ is a \emph{quantic nucleus},
if $j$ satisfies the following for each $a, b \in \mathcal{Q}$
\begin{itemize}
  \item $j$ is order-preserving,
  \item $a \leq j a$,
  \item $j j a \leq j a$,
  \item $j a \cdot j b \leq j (a \cdot b)$,
  \item $j \varepsilon \leq \varepsilon$.
\end{itemize}
An element $a \in \mathcal{Q}$ is \emph{$j$-closed} if $j a = a$. Let:
\begin{center}
$\mathcal{Q}_j = \{ a \in \mathcal{Q} \: | \: j a = a \}$
\end{center}
\end{definition}

\begin{prop} Let $j : \mathcal{Q} \to \mathcal{Q}$ be a closure operator on a 
  quantale $\mathcal{Q}$, then $j$ is a quantic nucleus iff
  for each $a, b \in \mathcal{Q}$ $j(a \cdot b) = j (j a \cdot j b)$.
\end{prop}

\begin{proof} Take any $a, b \in \mathcal{Q}$.
  Assume $j(a \cdot b) = j (j a \cdot j b)$,
  then 
  \begin{center}
  $ja \cdot jb \leq j (j a \cdot j b) = j (a \cdot b)$.
  \end{center}

  Assume $j$ is a quantic nucleus, then
  \begin{prooftree}
    \AxiomC{$a \leq j a$}
    \AxiomC{$b \leq j b$}
    \BinaryInfC{$a \cdot b \leq j a \cdot j b$}
    \UnaryInfC{$j(a \cdot b) \leq j (j a \cdot j b)$}
    \AxiomC{$j a \cdot j b \leq j (a \cdot b)$}
    \UnaryInfC{$j(j a \cdot j b) \leq j j (a \cdot b) = j (a \cdot b)$}
    \BinaryInfC{$j(a \cdot b) = j (a \cdot b)$}
  \end{prooftree}
\end{proof}


\begin{prop}
  $\mathcal{Q}_j$ forms a subquantale of $\mathcal{Q}$. Moreover, the map $a \mapsto j a$
  defined a surjective homomorphism from $\mathcal{Q}$ onto $\mathcal{Q}_j$.
\end{prop}

\begin{proof} See the complete proof in \cite[Theorem 3.1.1]{570645}.
Take any elements $a, b \in \mathcal{Q}_j$, any indexed family $\{ a_i \in \mathcal{Q}_j \: | \: i \in I \}$
and define the operations the following way:
\begin{itemize}
  \item $a \cdot_{\mathcal{Q}_i} b = j(a \cdot_{\mathcal{Q}} b)$
  \item ${\bigvee \limits_{i \in I}}_{\mathcal{Q}_j} a_i = j ({\bigvee \limits_{i \in I}}_{\mathcal{Q}} a_i)$
\end{itemize}
Note that the identity element is already closed by the definition of a quantic nucleus.
\end{proof}

Let $\mathcal{P} = (P, \leq)$ be a poset and $x \in \mathcal{P}$, the upper cone generated by $x$ is the set $\uparrow x = \{ y \in P \: | \: x \leq y \}$. 
Let $A \subseteq \mathcal{P}$, define $\uparrow A$ as
\begin{center}
$\uparrow A = \bigcup \limits_{x \in A} \uparrow x$
\end{center}
A subset set $A$ is upward closed whenever $\uparrow A = A$.
We say that $y$ \emph{refines} $x$ if $x \leq y$, or, equivalently, 
$\uparrow y \subseteq \uparrow x$. We say that a subset $Y$ \emph{refines} 
if $Y \subseteq \uparrow X$, that is, every element of $y$ refines some element of $X$.
The set $\operatorname{Up}(\mathcal{P})$ is the set of all upward closed subsets of $\mathcal{P}$.

\section{Monoidal Grothendieck Topologies and Noncommutative Sites}

Let $\mathcal{C}$ be a category and let $X \in \mathcal{C}$, then $\mathcal{C}^{(0)}/_X$ is the full category of
$\mathcal{C}/_X$ generated by those maps $U \to X$ which factor through some $V \in \mathcal{C}$.

A \emph{presheaf} is a functor $F : \mathcal{C}^{op} \to \operatorname{Set}$. 
The \emph{category of all presheaves} on $\mathcal{C}$ is the category of functors $\operatorname{PSh}(\mathcal{C}) = \operatorname{Set}^{\mathcal{C}^{op}}$ and their natural transformations.
The \emph{$\operatorname{Hom}$-functor} is a bifunctor $\operatorname{Hom} : \mathcal{C}^{op} \times \mathcal{C} \to \operatorname{Set}$ defined as:
  \begin{center}
    $\operatorname{Hom}_{\mathcal{C}} : (A, B) \mapsto \{ f \in \operatorname{Mor}(\mathcal{C}) \: | \: f : A \to B \}$
  \end{center}
With each $C \in \Ob{C}$ we can associate a presheaf ${\bf y}_C \in \operatorname{PSh}(\mathcal{C})$ on $\mathcal{C}$ defined as:
  \begin{center}
    ${\bf y}_C(D) = \operatorname{Hom}_{\mathcal{C}}(D, C)$ for $D \in \Ob{C}(\mathcal{C})$.

    ${\bf y}_C(f) : \operatorname{Hom}_{\mathcal{C}}(D, C) \to \operatorname{Hom}_{\mathcal{C}}(D', C)$
  \end{center}
 such that ${\bf y}_C(f)(g) = g \circ f$ for $f : D' \to D$ and $g : D \to C$.
Functors of the form of ${\bf y}_C$ (up to isomorphism) are called \emph{representable} functors.

The \emph{Yoneda embedding} is the following (full and faithful) functor from $\mathcal{C}$ to the category of presheaves:
\begin{center}
  ${\bf y} : \mathcal{C} \to \operatorname{Set}^{\mathcal{C}^{op}}$

  ${\bf y}(C) = \operatorname{Hom}_{\mathcal{C}}(-, C) = {\bf y}_C$
\end{center}

The following isomorphism
\begin{center}
  $\theta : \operatorname{Hom}_{\operatorname{PSh}(\mathcal{C})}({\bf y}_C, P) \cong P(C)$
\end{center}
is folklore.

We need some basics of coends and the Day convolution, the reader can find more details in 
\cite{loregian2021co}, \cite[Chapter 1]{riehl2014categorical} or \cite[Chapter 6]{borceux1994handbook}.

\begin{definition}[Coend]
  Let $P : \mathcal{C}^{\operatorname{op}} \times \mathcal{C} \to \operatorname{Set}$
  be a profunctor, the coend
    \[\int^{X \in \mathcal{C}} P(X, X)\]
    is a set with a family of arrows $i_X : P(X, X) \to \int^Z P(Z, Z)$
    for each $X \in \operatorname{Ob}(\mathcal{C})$ such that following square commutes for each $f : X \to Y$:

    \centerline{
      \xymatrix{
        H(Y, X) \ar[r]^{f_*} \ar[d]_{f^*} & H(Y, Y) \ar[d]^{i_Y} \\
        H(X, X) \ar[r]_{i_X} & \int^Z P(Z, Z)
      }
    }

\end{definition}

Equivalently, the coend $\int^Z P(Z,Z)$ can be equivalently defined as the coequaliser of the following diagram:

\centerline{
  \xymatrix{
    \bigsqcup \limits_{f} H(\operatorname{cod}(f), \operatorname{dom}(f)) \ar@<-.5ex>[r] \ar@<.5ex>[r] & \bigsqcup \limits_{X \in \operatorname{Ob}(\mathcal{C})} H(X, X) \ar@{-->}[r] & \int^Z H(Z, Z)
  }
}

\begin{definition}[Day convolution]
  Let $(\mathcal{C}, \otimes, \mathds{1})$ be a monoidal category and let
  $F, G : \mathcal{C}^{\operatorname{op}} \to \operatorname{Set}$ be presheaves. 
  The Day convolution of $F$ and $G$ is the following coend, for each $X \in \Ob{C}$:
    \[
    (F \star G)(X) := \int^{X_1, X_2} \operatorname{Hom}_{\mathcal{C}}(X, X_1 \otimes X_2) \times F(X_1) \times F(X_2).
    \]
\end{definition}

The following is due to Day, see \cite{day1970construction}.

\begin{prop}
  Let $(\mathcal{C}, \otimes, \mathds{1})$ be a monoidal category, then:
  \begin{enumerate}
    \item $(\operatorname{PSh}(\mathcal{C}), \star, {\bf y}(\mathds{1}))$ is a monoidal category,
    \item $\mathcal{C}$ embeds into $(\operatorname{PSh}(\mathcal{C}), \star, {\bf y}(\mathds{1}))$
    by the Yoneda embedding.
  \end{enumerate}
\end{prop}

\begin{definition}[Monoidal Localisation]
\end{definition}

\begin{definition}
Let $\mathcal{C}$ be a category, a collection of morphisms $S \subseteq \operatorname{Hom}(\mathcal{C})$ with a common
codomain $X \in \Ob{C}$ is called a \emph{sieve on $X$} if $f \in S$ implies $g \circ f \in S$ whenever such a composition is well-defined.
Let $S$ be a sieve on $X$ and $f : Y \to X$ be an arrow, then the set
\begin{center}
  $f^*(S) = \{ g \in \operatorname{Hom}(\mathcal{C}) \: | \: g \circ f \in S \}$
\end{center}
is a sieve on $D$.

Let $X \in \Ob{C}$, then \emph{the maximal sieve} generated by $X$ is the following collection:
\begin{center}
  $t_X = \{ f \: | \: \operatorname{cod}(f) = X \}$,
\end{center}
\end{definition}

\begin{definition}
  Let $\{ f_i : U_i \to U \: \}_{i \in I}$ be a family of arrows in a category $\mathcal{C}$.
  A \emph{refinement} $\{ f_j : U_j \to U \}_{j \in J}$ is a family of arrows in $\mathcal{C}$
  if for each $j \in J$ there is $i \in I$ such that $f_j :  U_j \to U$ factors through $f_i : U_i \to U$:
  
  \centerline{
    \xymatrix{
    U_j \ar[rr]^{f_j} \ar@{-->}[dr] && U \\
    & U_i \ar[ur]_{f_i}
  }}
\end{definition}

There is also a more general definition of refinement for different codomains.
\begin{definition}
  Let $U = \{ U_i \to X \}_{i \in I}$ be a family of morphisms in $\mathcal{C}$,
  then $V = \{ U_j \to Y\}_{j \in J}$ is a refinement of $U$ is a family of morphisms
  such that for each $j \in J$ there is $i \in I$ such that the following square commutes


  \centerline{
    \xymatrix{
    U_j \ar[r]^{f_j} \ar[d] & Y \ar[d] \\
    U_i \ar[r]_{f_i} & X
  }
  }
\end{definition} 

\begin{definition}
  Let $(\mathcal{C}, \otimes, \mathds{1})$ be a monoidal category, a \emph{monoidal Grothendieck topology}
  is a function $\operatorname{Cov}$ mapping each $X \in \operatorname{Hom}(\mathcal{C})$ to some collections of 
  sieves $\{f : U_i \to X \}_{i \in I}$ called \emph{coverings sieves of $X$}. 
  $\operatorname{Cov}$ satisfies the following properties for each $X, Y \in \Ob{C}$:
  \begin{enumerate}
    \item (\emph{Existence}) $t_X \in \operatorname{Cov}(X)$.
    \item (\emph{Transitivity}) Let $S_1 \in \operatorname{Cov}(X)$ and let $S_2$
    be a sieve on $X$ such that $f^*(S_2) \in \operatorname{Cov}(Y)$ for any $f : Y \to X$ in $S_1$, 
    then $S_2 \in \operatorname{Cov}(X)$.
    \item (\emph{Stability}) If $S \in \operatorname{Cov}(X)$, then for any arrow $f : X \to Y$ one
    $f^*(S) \in \operatorname{Cov}(Y)$.
    \item (\emph{Multiplicativity}) If $S_1 \in \operatorname{Cov}(X)$
    and $S_2 \in \operatorname{Cov}(Y)$, then there exists a covering sieve $S \in \operatorname{Cov}(X \otimes Y)$
    that refines $S_1 \otimes S_2$.
    \item (\emph{Identity axiom}) $\operatorname{Cov}(\mathds{1}) = t_{\mathds{1}}$.
  \end{enumerate}
\end{definition}

A monoidal Grothendieck topology is \emph{strong} if it satisfies the following extra-property:
let $S_1, S_2$ be family of arrows and let $X \in \Ob{C}$ such that there exists $S \in \operatorname{Cov}(X)$
such that $S$ refines $S_1 \otimes S_2$, then there
exist $X_1, X_2 \in \Ob{C}$ and $f : X_1 \otimes X_2 \to X$
such that there are $S_i' \subseteq S_i$ and $S_i' \in \operatorname{Cov}(X_i)$
for $i \in \{ 1, 2\}$.

\begin{prop} Let $(\mathcal{C}, \otimes, \mathds{1})$ be a monoidal category and let
  $\operatorname{Cov} : \operatorname{Hom}(\mathcal{C}) \to \operatorname{Set}$ be a function mapping each object to 
  some collection of sieves, then the following are equivalent:
  \begin{enumerate}
    \item $\operatorname{Cov}$ is a monoidal Grothendieck topology,
    \item $\operatorname{Cov}$ satisfies the axioms of Grothendieck topology, the identity axiom and the following property:
    if $X \in \Ob{C}$ and $S \in \operatorname{Cov}(X)$, then
    if $Y \in \Ob{C}$, then $Y \otimes S \in \operatorname{Cov}(Y \otimes X)$
    and $S \otimes Y \in \operatorname{Cov}(X \otimes Y)$.
  \end{enumerate}
\end{prop}

\begin{proof}
$ $
  \begin{enumerate}
    \item Assume $\operatorname{Cov}$ defines a monoidal Grothendieck topology. 
    Let $X, Y \in \operatorname{Cov}(\mathcal{C})$ and let $S = \{ f_i : U_i \to X \}_{i \in I} \in \operatorname{Cov}(X)$, then
    Consider
    \begin{center}
    $S \otimes Y = \{ f_i \otimes {\bf id}_Y : U_i \otimes Y \to X \otimes Y \}$
    \end{center}
    then it can be refined to some cover of $X \otimes Y$ by the multiplicativity axiom.
    \item Assume $\operatorname{Cov}$ defines a Grothendieck topology on $\mathcal{C}$
    satisfying the identity axiom and the functors $Y \otimes \underline{\:\:\:}$
    and $\underline{\:\:\:} \otimes Y$ commute with $\operatorname{Cov}$.
    Take $\{ f_i : U_i \to X \}_{i \in I} \in \operatorname{Cov}(X)$ and $\{ f_j : U_j \to Y \}_{j \in J} \in \operatorname{Cov}(Y)$,
      then, in particular one has
      \begin{center}
        $S = \{ f_i \otimes {\bf id}_Y : U_i \otimes Y \to X \otimes Y \} \in \operatorname{Cov}(X \otimes Y)$
      \end{center}
  
      Fix any $i \in I$ and $j \in J$ and take $f_i \otimes f_j : U_i \otimes U_j \to X \otimes Y$. The triangle
      following triangle

      \centerline{
        \xymatrix{
          U_i \otimes U_j \ar[rr]^{f_i \otimes f_j} \ar[dr]_{{\bf id}_{U_i} \otimes f_j} && X \otimes Y \\
          & U_i \otimes Y \ar[ur]_{f_i \otimes {\bf id}_Y}
        }
      }
      obviously commutes, so $S$ refines $\{ f_i \otimes f_j\}_{i \in I, j \in I}$.
  \end{enumerate}
\end{proof}

\begin{definition}
Let $(\mathcal{C}, \otimes, \mathds{1})$ be a monoidal category with a monoidal Grothendieck topology $\operatorname{Cov}$.
A \emph{sheaf} is a presheaf $F : \mathcal{C}^{op} \to \operatorname{Set}$ such that for every 
$C \in \Ob{C}$, for every covering sieve $S \in \operatorname{Cov}(C)$ and for every family of the form
\begin{center}
  $\{ x_f \in F(\operatorname{dom}(f)) \: | \: f \in S \}$
\end{center}
such that
\begin{center}
  $F(g)(x_f) = x_{g \circ f}$
\end{center}
for each $f \in S$ and for each arrow $g$ composable with $f$,
there exists a unique $x \in F(C)$ such that $x_f = P(f)(x)$
for each $f \in S$.

The category of monoidal sheaves ${\bf MSh}(\mathcal{C}, \operatorname{Cov})$
is the category of sheaves $P : \mathcal{C}^{op} \to \operatorname{Set}$ with natural transformations
as morphisms.
\end{definition}

\begin{prop}
  Let $(\mathcal{C}, \otimes, \mathds{1}, \operatorname{Cov})$ be a noncommutative site and
  let $F : \mathcal{C}^{\operatorname{op}} \to \operatorname{Set}$ , then the following are
  equivalent:
  \begin{enumerate}
    \item $F$ is a sheaf,
    \item For every covering $\{ U_i \to X \}_{i \in I}$, the following diagram

    \centerline{
      \xymatrix{
        F(X) \ar[r] & \prod \limits_{i \in I} F(U_i) \ar@<-.5ex>[r] \ar@<.5ex>[r] & \prod \limits_{i, j \in I} F(U_i \times_X U_j)
      }
    }
    is an equaliser.
    \item For every covering sieve $\mathcal{C}^{0}/X \subseteq \mathcal{C}/X$
    the map $F(X) \to \colim_{U \in \mathcal{C}^{0}/X} F(U)$ is a bijection.
  \end{enumerate}
\end{prop}

\begin{definition}
  Let $(\mathcal{C}, \otimes, \mathds{1})$ be a monoidal category, then $\mathcal{C}$
  is a \emph{monoidal topos} if the following axioms are satisfied:
  \begin{enumerate}
    \item $\mathcal{C}$ admits all coproducts and all of them are disjoint,
    \item Effective epimorphisms are closed under pullback,
    \item Let $f : X \to Y$ in $\mathcal{C}$, the pullback functor
    $f^* : \mathcal{C}/_Y \to \mathcal{C}/_X$ preserves all coproducts,
    \item Every equivalence relation is effective,
    \item There exists a set $U \subseteq \Ob{C}$ such that
    for each $X \in \Ob{C}$ there exists a covering
    $\{ U_i \to X \}_{i \in I}$
    where for each $i \in I$ each $U_i \in U$. That is, a covering $\{ U_i \to X \}_{i \in I}$
    is generated by $U$.
    \item For $X \in \Ob{C}$ and for any indexed family 
    $\{ X_i \in \Ob{C} \: | \: i \in I \}$ for any index set $I \neq \emptyset$, then 
    there are natural isomorphisms: 
    \begin{center}
    $X \otimes (\colim \limits_{i \in I} X_i) \cong (\colim \limits_{i \in I} (X \otimes X_i))$,

    $(\colim \limits_{i \in I} X_i) \otimes X \cong (\colim \limits_{i \in I} (X_i \otimes X))$.
    \end{center}
  \end{enumerate}
\end{definition}

\section{Sheafification and Giraud's Theorem}

Fix a category $\mathcal{X}$ such that there is an equivalence $\mathcal{X} \simeq \operatorname{MShv}(\mathcal{C})$
for some monoidal category $\mathcal{C}$.

\subsection{Sheafification}

First of all, we show the following fact:
\begin{theorem}~\label{adjoint}
  Let $(\mathcal{C}, \otimes, \mathds{1}, \operatorname{Cov})$ be a non-commutative site,
  then the inclusion function $F : \operatorname{MShv}(\mathcal{C}) \hookrightarrow \operatorname{PSh}(\mathcal{C})$
  has a left adjoint $L : \operatorname{PSh}(\mathcal{C}) \to \operatorname{MShv}(\mathcal{C})$.
  Moreover $L$ is a monoidal functor.
\end{theorem}

Given a presheaf $P$, $L P$ is the \emph{sheafification} of $P$.

\begin{proof}
First of all, let us describe the sheafification procedure explicitly.
Let $F : \mathcal{C}^{\operatorname{op}} \to \operatorname{Set}$ be a presheaf, for $X \in \Ob{C}$ we let
\begin{center}
  $F^*(X) := \colim \limits_{\mathcal{C}^{0}/X} \lim \limits_{U \in \mathcal{C}^{0}/X} F(U)$
\end{center}
As far as $\mathcal{C}/X$ is always a covering, we have the following canonical map:

\centerline{
  \xymatrix{
  F(X) \cong \lim \limits_{U \in \mathcal{C}^{0}/X} F(U) \ar[rr]^{\alpha_F} && \colim \limits_{\mathcal{C}^{0}/X} \lim \limits_{U \in \mathcal{C}^{0}/X} \cong F^{*}(X)
}}

We conclude the theorem from the following claim:
\begin{claim} Let $F : \mathcal{C}^{\operatorname{op}} \to \operatorname{Set}$ be a presheaf, then the following holds:

  \begin{enumerate}
    \item The composite map of the form

    \centerline{
      \xymatrix{
        F \ar[r]^{\alpha_F} & F^* \ar[r]^{\alpha_{F^*}} & F^{**}
      }
    }
    is a sheafification of $F$. Moreover, the following canonical map is a bijection for every sheaf 
    $G : \mathcal{C}^{\operatorname{op}} \to \operatorname{Set}$

    \centerline{
      \xymatrix{
        \operatorname{Hom}_{\operatorname{Shv}(\mathcal{C})}(F^**, G) \ar[r] & \operatorname{Hom}_{\operatorname{PSh}(\mathcal{C})}(F, G).
      }
    }
    \item The functor $F \mapsto F^{*}$ is left exact and monoidal.
  \end{enumerate}
\end{claim}

\begin{proof} 
$ $

  \begin{enumerate}
    \item Let us define a category $\operatorname{Cov}(\mathcal{C})$ the following way:
    \begin{itemize}
      \item $\operatorname{Ob}(\operatorname{Cov}(\mathcal{C}))$ consists of pairs
      $(X, \mathcal{C}^{0}/X)$ where $X \in \mathcal{C}$ and $\mathcal{C}^{0}/X \subseteq \mathcal{C}/X$ is a covering sieve.
      \item Given $(X, \mathcal{C}^{0}/X)$ and $(Y, \mathcal{C}^{0}/Y)$, a
      morphism $(X, \mathcal{C}^{0}/X) \to (Y, \mathcal{C}^{0}/Y)$ is a morphism $f : X \to Y$
      in $\mathcal{C}$ such that if a morphism $g : U \to X$ belongs to $(X, \mathcal{C}^{0}/X)$, 
      then $f \circ g$ belongs to $(Y, \mathcal{C}^{0}/Y)$.
    \end{itemize}

    Let us observe that $\operatorname{Cov}(\mathcal{C})$ is a monoidal category indeed, let
    \begin{center}
      $(X, \mathcal{C}^{0}/X) \otimes (Y, \mathcal{C}^{0}/Y) := (X \otimes Y, \mathcal{C}^{0}/(X \otimes Y))$.
    \end{center}
    Note that such a product is well-defined since $\mathcal{C}^{0}/(X \otimes Y)$ is a covering sieve by the multiplicativity axiom.
    Further, observe that we have functors $i : \mathcal{C} \to \operatorname{Cov}(\mathcal{C})$ and $\rho : \operatorname{Cov}(\mathcal{C}) \to \mathcal{C}$
    realised as
    \begin{center}
      $i : X \mapsto (X, X, \mathcal{C}^{0}/X)$

      $\rho : (X, X, \mathcal{C}^{0}/X) \mapsto X$
    \end{center}
 
    
    \item To show that the functor is (lax) monoidal, let 
    $F, G : \mathcal{C}^{\operatorname{op}} \to \operatorname{Set}$, then
    
    \begin{multline*}
      (F^* \star G^*)(X) = \\
      \int^{X_1, X_2} \operatorname{Hom}_{\mathcal{C}}(X, X_1 \otimes X_2) \times F^*(X_1) \times G^*(X_2) = \\
      \int^{X_1, X_2} \operatorname{Hom}_{\mathcal{C}}(X, X_1 \otimes X_2) \times \colim \limits_{\mathcal{C}^{0}/X_1} \lim \limits_{U_1} F(U_1) \times \colim \limits_{\mathcal{C}^{0}/X_2} \lim \limits_{U_2} G(U_2) \cong \\
      \int^{X_1, X_2} \operatorname{Hom}_{\mathcal{C}}(X, X_1 \otimes X_2) \times \colim \limits_{\mathcal{C}^{0}/_{X_1}, \mathcal{C}^{0}/_{X_2}} (\lim \limits_{U_1} F(U_1) \times \lim \limits_{U_2} G(U_2)) \cong \\
      \int^{X_1, X_2} \operatorname{Hom}_{\mathcal{C}}(X, X_1 \otimes X_2) \times \colim \limits_{\mathcal{C}^{0}/_{X_1}, \mathcal{C}^{0}/_{X_2}} \lim \limits_{U_1, U_2} (F(U_1) \times G(U_2)) \cong \\
      \colim \limits_{\mathcal{C}^{0}/_{X_1}, \mathcal{C}^{0}/_{X_2}} \lim \limits_{U_1, U_2} \int^{X_1, X_2} \operatorname{Hom}_{\mathcal{C}}(X, X_1 \otimes X_2) \times F(U_1) \times G(U_2) \cong_? \\
      \colim \limits_{\mathcal{C}^{0}/_{X}} \lim \limits_{U} (F \star G)(U) = (F \star G)^*(X). \\
     \end{multline*}
  \end{enumerate}
\end{proof}
\end{proof}

\subsection{Giraud's axioms}

Further let us show the following:
\begin{theorem} $\mathcal{X}$ satisfies:
  \begin{enumerate}
    \item $\mathcal{X}$ has arbitrary colimits and limits.
    \item Every equivalence relation in $\mathcal{X}$ is effective.
    \item Coproducts in $\mathcal{X}$ are disjoint.
    \item Tensors are preserved under arbitrary colimits.
    \item 
  \end{enumerate}
\end{theorem}

Before proving theorem~\ref{adjoint}, we show the following lemmas.

\begin{lemma}
  $\mathcal{X}$ has arbitrary colimits and limits.
\end{lemma}

\begin{proof}
$\mathcal{X}$ has colimits since $\operatorname{PSh}(\mathcal{C})$ has colimits computed pointwise for $X \in \Ob{C}$:
\begin{center}
  $(\colim F_{\alpha})(X) = \colim (F_{\alpha}(X))$
\end{center}
\end{proof}

\begin{lemma}
  Every equivalence relation in $\mathcal{X}$ is effective.
\end{lemma}

\begin{proof}

\end{proof}

\begin{lemma}
  Coproducts in $\mathcal{X}$ are disjoint.
\end{lemma}

\begin{proof}

\end{proof}

\begin{lemma}
  Colimits in $\mathcal{X}$ are universal and there are canonical isomorphisms:
  \begin{center}
  $\colim \limits_i (X \otimes X_i) \to X \otimes \colim \limits_i X_i$

  $\colim \limits_i (X_i \otimes X) \to \colim \limits_i X_i \otimes X$
  \end{center}
\end{lemma}

\begin{proof}

\end{proof}

\section{Proof of Theorem~\ref{adjoint}}

  Fix a non-commutative site $(\mathcal{C}, \otimes, \mathds{1}, \operatorname{Cov})$.

  Let $F : \mathcal{C}^{\operatorname{op}} \to \operatorname{Set}$ be a sheaf and let 
  $X \in \Ob{C}$, then we call $x \in F(X)$
  a \emph{section of $F$ over $X$}.

\begin{definition}
  Let $X \in \Ob{C}$, a \emph{sieve} on $X$ is a full subcategory 
  $\mathcal{C}^{(0)}/_X \subseteq \mathcal{C}/_X$ with the following property:
  if $U \to V \in \mathcal{C}/_X$ and $V \in \mathcal{C}^{(0)}/_X$, then $U \in \mathcal{C}^{(0)}/_X$.

  A sieve $\mathcal{C}^{(0)}$ is a covering if it contains $\{ U_i \to X \} \in \operatorname{Cov}(X)$.
\end{definition}

\begin{prop}~\label{covering:sieve:prop1}
  Let $\{ f_i : U_i \to X \}$ be a family of morphisms with a common codomain $X$
  and let $\mathcal{C}^{(0)}/_X$ be a sieve generated by those maps $U \to V$ that factor through some $f_i$.
  Then $\mathcal{C}^{(0)}/_X$ is a covering sieve if and only if $\{ f_i : U_i \to X \} \in \operatorname{Cov}(X)$.
\end{prop}

\begin{proof}

\end{proof}

\begin{prop}
  Let $\{ f_i : U_i \to X \}$ be a family of morphisms and let $\mathcal{C}^{(0)}/_X$ be a sieve
  as in Proposition~\ref{covering:sieve:prop1}. Let $Y \in \Ob{C}$, then
  $Y \otimes {\mathcal{C}^{(0)}/_X}$ and ${\mathcal{C}^{(0)}/_X} \otimes Y$ are covering sieves.
\end{prop}

\begin{proof}

\end{proof}

\section{Giraud's Theorem}

\begin{theorem}

  Let $(\mathcal{C}, \otimes, \mathds{1})$ be a monoidal category, then the following statements are equivalent:
  \begin{enumerate}
    \item $\mathcal{C}$ is equivalent to the category of sheaves with a monoidal Grothendieck topology,
    \item There exists a small monoidal category $\mathcal{D}$ such that there is a fully faithful embedding
    $F : \mathcal{C} \hookrightarrow \operatorname{PSh}(\mathcal{D})$ with the Day convolution monoidal 
    structure such that there is $F^* : \operatorname{PSh}(\mathcal{D}) \to \mathcal{C}$
    such that $F^{*} \dashv F$ and $F^*$ is a monoidal functor preserving finite limits.
    \item $\mathcal{C}$ is a monoidal topos.
  \end{enumerate}

\end{theorem}

\begin{proof}
  $ $

  \begin{enumerate}
    \item $(1) \Rightarrow (2)$
    \item $(2) \Rightarrow (3)$
    \item $(3) \Rightarrow (1)$
  \end{enumerate}
\end{proof}


The idea of the above lemma is adapted from, for example, \cite[Theorem 1.1.13]{caramello2018theories}.
\begin{prop}
  Let $\mathcal{C}$ be a monoidal topos, let $C, D \in \Ob{C}$
  and let $f : C \to D$, then:
  \begin{enumerate}
  \item $\operatorname{Sub}(C)$ forms a quantale.
  \item The pullback function $f^* : \operatorname{Sub}(D) \to \operatorname{Sub}(C)$
  has a left adjoint $\exists_f : \operatorname{Sub}(C) \to \operatorname{Sub}(D)$
  and a right adjoint $\forall_f : \operatorname{Sub}(C) \to \operatorname{Sub}(D)$.
  \end{enumerate}
\end{prop}

\begin{proof}
\begin{enumerate}
\item 
\begin{enumerate}
  \item $\operatorname{Sub}(C)$ is a sup-lattice.
  \item $(\operatorname{Sub}(\mathcal{C}), \cdot, \varepsilon)$ is a monoid.
  \item The monoidal distributivity law.
\end{enumerate}
\item 
\end{enumerate}
\end{proof}

\begin{definition}
  Let $\mathcal{C}$ be a monoidal topos with a subobject classifier 
  $\Omega \in \Ob{C}$, then a \emph{noncommutative Lawvere-Tierney topology} is an arrow $j : \Omega \to \Omega$ 
  satisfying the following axioms:
  \begin{enumerate}
    \item $j = j \circ \operatorname{true}$:

    \centerline{
      \xymatrix{
      \top \ar[r]^{\operatorname{true}} \ar[dr]_{\operatorname{true}} & \Omega \ar[d]^j \\
      & \Omega
      }
    }
    \item $j \circ j = j$:

    \centerline{
      \xymatrix{
        \Omega \ar[r]^j \ar[dr]_j & \Omega \ar[d]^j \\
        & \Omega
      }
    }
    \item $j \circ \cdot = j \circ \cdot \circ j \otimes j$:

    \centerline{
      \xymatrix{
        & \Omega \otimes \Omega \ar[dr]^{\cdot} & \\
      \Omega \otimes \Omega \ar[ur]^{j \otimes j} \ar[d]_{\cdot} && \Omega \ar[d]^{j} \\
      \Omega \ar[rr]_{j} && \Omega
      }
    }
    \item $j \circ \varepsilon = \varepsilon$:

    \centerline{
      \xymatrix{
        \mathds{1} \ar[r]^{\varepsilon} \ar[dr]_{\varepsilon} & \Omega \ar[d]^j \\
        & \Omega
      }
    }
  \end{enumerate}
\end{definition}

A \emph{noncommutative site} is a monoidal topos $\mathcal{C}$ equipped with
a noncommutative Lawvere-Tierney topology.

\begin{prop}
  Let $\mathcal{C}$ be a monoidal category,
  then $(\operatorname{Set}^{\mathcal{C}^{\operatorname{op}}}, \star, {\bf y}(\mathds{1}))$
  has a monoidal subobject classifier.
\end{prop}

\begin{proof}
  As usual, let
  \begin{center}
    $\Omega(X) = \{ S \: | \: \text{$S$ is a sieve on $X$ in $\mathcal{C}$ }\}$
  \end{center}
  for $X \in \Ob{C}$. Let $f : Y \to X$ be a morphism, then the morphism 
  $\Omega(f) : \Omega(X) \to \Omega(Y)$ is defined as follows, for $S$, a sieve on $X$:
  \begin{center}
    $\Omega(f)(S) = \{ g \: | \: g \circ f \in S \}$.
  \end{center}
\end{proof}

\begin{theorem}
  Let $(\mathcal{C}, \otimes, I)$ be a monoidal category, then 
  $(\operatorname{Set}^{\mathcal{C}^{\operatorname{op}}}, \star, {\bf y}(\mathds{1}))$ is a monoidal topos.
\end{theorem}

\begin{proof}

\end{proof}

The following is a categorical generalisation of \cite[Theorem 5]{goldblatt2006kripke}.
One can think of it as a noncommutative generalisation of \cite[§V.1, Theorem 2]{maclane2012sheaves}.
\begin{theorem}
  Let $(\mathcal{C}, \otimes, \mathds{1})$ be a monoidal category with a monoidal Grothendieck topology $\operatorname{Cov}$,
  then $\operatorname{Cov}$ determines a noncommutative Lawvere-Tierney topology on $\operatorname{Set}^{\mathcal{C}^{\operatorname{op}}}$.
\end{theorem}

\begin{proof}

\end{proof}

Moreover
\begin{theorem}
  Let $(\mathcal{C}, \otimes, \mathds{1})$ be a monoidal category, then there is a bijection
  between noncommutative Grothendieck topologies on $\mathcal{C}$ and
  noncommutative Lawvere-Tierney topologies on the monoidal topos $(\operatorname{Set}^{\mathcal{C}^{\operatorname{op}}}, \star, {\bf y}(\mathds{1}))$.
\end{theorem}

\section{Infinitary Substructural Logic}

\begin{definition}
A first-order signature is a triple $\Omega = (\operatorname{Sort}, \operatorname{Fn}, \operatorname{Rel})$
where
\begin{itemize}
  \item $\operatorname{Sort}$ is a set of \emph{sorts},
  \item $\operatorname{Fn}$ is a set of \emph{function symbols}. We associate a type with every $f \in \operatorname{Fn}$ written as
  \begin{center}
    $f : A_1, A_2, \ldots, A_n \to A$
  \end{center}
  where $A_1, A_2, \ldots, A_n, A \in \operatorname{Sort}$,
\end{itemize}
  \item $\operatorname{Rel}$ is a set of relation symbols. As above, we associate a type with every $R \in \operatorname{Rel}$:
  \begin{center}
    $R \hookrightarrow A_1, A_2, \ldots, A_n$.
  \end{center}
\end{definition}

We associate the set of individual variables $\{ v_n : A \: | \: n < \omega\}$ 
with each sort $A \in \operatorname{Sort}$, so we define terms the following way:
\begin{definition}
  The collection of terms over a signature $\Sigma$ is defined inductively:
  \begin{itemize}
    \item Every variable $v : A$ is a term of sort $A$,
    \item Let $t_1 : A_1, \ldots, t_n : A_n$ be $\Sigma$-terms and let
    $f : A_1, A_2, \ldots, A_n \to A$ be a function symbol, then 
    $f(t_1, \ldots, t_n)$ is a term of sort $A$.
  \end{itemize}
\end{definition}

Let $t$ be a term, then the set of free variables $\operatorname{FV}$ is defined by induction on $t$:

\begin{center}
$\begin{array}{lll}
  & \operatorname{FV}(v : A) = \{ v : A \} & \\
  & \operatorname{FV}(f(t_1, \ldots, t_n)) = \bigcup \limits_{1 \leq k \leq n} \operatorname{FV}(t_k)&
\end{array}$
\end{center}

\begin{definition}
  Let $\Sigma$ be a first-order signature,
  the collection of atomic formulas $\operatorname{At}(\Sigma)$ is defined as follows.
  Let $R \hookrightarrow A_1, \ldots, A_n$ be a relation symbol
  and let $t_1 : A_1, \ldots, t_n : A_n$ be terms of the corresponding sorts, then
  \begin{center}
    $R(t_1, \ldots, t_n)$
  \end{center}
  is an atomic formula. The set of free variables of an atomic formula is defined as
  \begin{center}
    $\operatorname{FV}(R(t_1, \ldots, t_n)) = \bigcup \limits_{1 \leq k \leq n} \operatorname{FV}(t_k).$
  \end{center}
\end{definition}

\begin{definition}
  Let us define a class $F$ of formulas over a signature $\Sigma$ is defined by joint induction
  with the corresponding finite sets of free variables:
  \begin{enumerate}
    \item (\emph{Truth}): $\top \in F$ with $\operatorname{FV}(\top) = \emptyset$,
    \item (\emph{Falsity}): $\bot \in F$ with $\operatorname{FV}(\bot) = \emptyset$,
    \item (\emph{Identity}): ${\bf 1} \in F$ with $\operatorname{FV}({\bf 1}) = \emptyset$,
    \item (\emph{Fusion}): if $\varphi, \psi \in F$, then $\varphi \bullet \psi \in F$ with 
    $\operatorname{FV}(\varphi \bullet \psi) = \operatorname{FV}(\varphi) \cup \operatorname{FV}(\psi)$,
    \item (\emph{Residuals}): if $\varphi, \psi \in F$, then $\varphi \setminus \psi, \varphi / \psi \in F$ and
    $\operatorname{FV}(\varphi \setminus \psi) = \operatorname{FV}(\psi / \varphi) = \operatorname{FV}(\varphi) \cup \operatorname{FV}(\psi)$.
    \item (\emph{Universal Quantifier}) Let $v_i$ be a variable and let $\varphi \in F$, then $\forall v_i \varphi$
    and $\operatorname{FV}(\forall v_i \varphi) = \operatorname{FV}(\varphi) - \{v_i\}$.
    \item (\emph{Existential Quantifier}) Let $v_i$ be a variable and let $\varphi \in F$, then $\exists v_i \varphi$
    and $\operatorname{FV}(\exists v_i \varphi) = \operatorname{FV}(\varphi) - \{v_i\}$.
    \item (\emph{Infinitary Disjunction and Infinitary Conjunction})
    Let $\{ \varphi_i \: | \: i \in I \}$ be an indexed set of formulas such that 
    $|\cup_{i \in I} \operatorname{FV}(\varphi_i)| < \omega$, then
    \begin{center}
      $\bigvee \limits_{i \in I} \varphi_i, \bigwedge \limits_{i \in I} \varphi_i \in F$
    \end{center}
  \end{enumerate}
\end{definition}

Fix a monoidal topos $\mathcal{C}$. Let $\alpha : U \to X$ be a generalised element with 
$\operatorname{Im}\alpha \in \operatorname{Sub}(X)$, let
\begin{center}
  $U \Vdash \varphi(\alpha)$ iff $\operatorname{Im}\alpha \leq \{ x \: | \: \varphi(x) \}$.
\end{center}

TODO: draw a proper diagram.

\begin{prop} The following holds:

  \begin{enumerate}
    \item (\emph{Monotonicity}) If $U \Vdash \varphi(\alpha)$, then for every $f : U' \to U$ in $\mathcal{C}$,
    then $U' \Vdash \varphi(\alpha \circ f)$.
    \item (\emph{Local character}) If $f : U' \twoheadrightarrow U$ and $U' \Vdash \varphi(f \circ \alpha)$,
    then $U \Vdash \varphi(\alpha)$.
  \end{enumerate}
\end{prop}

\begin{theorem}
  Let $X \in \Ob{X}$ and $\alpha : U \to X$ a generalised element of $X$.
  Let $\varphi(x), \psi(x)$ be formulas with a free variable $x$ of sort $X$,
  then
  \begin{enumerate}
    \item $U \Vdash {\bf 1}$ iff ${\bf 1}$ ...
    \item $U \Vdash \bot$ iff $X = \colim \emptyset$.
    \item $U \Vdash \top$ iff $X = \lim \emptyset$
    \item $U \Vdash (\varphi \bullet \psi)(\alpha)$ iff there $U_1, U_2 \in \Ob{C}$ such that there is arrow
    $f : U \to U_1 \otimes U_2$ such that $U_1 \Vdash \varphi(\alpha)$ and $U_2 \Vdash \varphi(\alpha)$. TODO: probably wrong.
    \item $U \Vdash (\varphi \setminus \psi)(\alpha)$ iff $U_1 \Vdash \varphi(\alpha)$ implies $U_1 \otimes U \Vdash \psi(\alpha)$.
    \item $U \Vdash (\psi / \varphi)(\alpha)$ iff $U_1 \Vdash \varphi(\alpha)$ implies $U \otimes U_1 \Vdash \psi(\alpha)$.
    \item $U \Vdash (\bigvee \limits_i \varphi)(\alpha)$ iff there exists $\{ f_i : U_i \to U \}_{i \in I} \in \operatorname{Cov}(U)$
    such that $\bigsqcup \limits_i U_i \twoheadrightarrow U$ is epic
    and for each $i \in I$ one has $U_i \Vdash \varphi_i(\alpha \circ f_i)$.
    \item $U \Vdash (\bigwedge \limits_i \varphi_i)(\alpha)$ iff for each $U \Vdash \varphi_i(\alpha)$.
    \item $U \Vdash \exists y \varphi(y, \alpha)$ iff there is $\{ U_i \to U \} \in \operatorname{Cov}(U)$
    and there is a generalised element $\beta : \bigsqcup \limits_i U_i \to Y$ ...
    \item $U \Vdash \forall y \varphi(y, \alpha)$ iff for every $V \in \Ob{C}$ and $p : V \to U$
    and for every generalised element $\beta : V \to Y$ such that $V \Vdash \varphi(p \circ \alpha, \beta)$.
    \item $U \Vdash \varphi \Rightarrow \psi$ iff $U \Vdash \varphi$ implies $U \Vdash \psi$.
  \end{enumerate}
\end{theorem}

\section{Completeness via Morleyisation}


\section{On Noncommutative Geometric Logic}

\subsection{One-sorted Version}


Let $\{ v_i \: | \: i < \omega \}$ be a set of individual variables and let 
$\{ P^k_i \: | \: k, i < \omega \}$ be a set of predicate letters where upper indices
are the corresponding arities.
The grammar of \emph{atomic formulas} is the set $At$ of all words of the form
\begin{center}
  $P^k_i(v_{n_1}, \ldots, v_{n_k})$
\end{center}
where $v_{n_1}, \ldots, v_{n_k}$ are individual variables and $P^{k}_i$ is a predicate letter of arity $k$.
A \emph{preformula} is an expression of one of the following form:
\begin{itemize}
\item Every atomic formula is a preformula,
\item ${\bf 1}$ is a preformula,
\item If $\varphi$ and $\psi$ are preformulas, so is $\varphi \bullet \psi$,
\item Let $\Phi$ be \emph{any} set of preformulas, then $\bigvee \Phi$ is a preformula,
\item Let $v$ be an individual variable and let $\varphi$ be a preformula, then $\exists v \varphi$,
\item Nothing else is a preformula.
\end{itemize}
Such definitions as free and bound variables are standard.

A \emph{formula} is a preformula with finitely many free variables.
\begin{definition}
\emph{Noncommutative geometric logic} consists of pairs of formulas $\varphi \Rightarrow \psi$ called \emph{sequents},
where $\Rightarrow$ is the metaimplication sign defined with the following axiom schemes and inference rules:

\vspace{\baselineskip}

\begin{minipage}{0.5\textwidth}
  \begin{flushleft}
    \begin{itemize}
      \item $\varphi \Rightarrow \varphi$,
      \item $(\varphi \bullet \psi) \bullet \theta \Leftrightarrow \varphi \bullet (\psi \bullet \theta)$,
      \item $\psi \bullet \bigvee \limits_{\varphi \in \Phi} \varphi \Leftrightarrow \bigvee \limits_{\varphi \in \Phi} (\psi \bullet \varphi)$,
      \item $\varphi \Leftrightarrow \varphi \bullet {\bf 1} \Leftrightarrow {\bf 1} \bullet \varphi$,
      \item $\psi \bullet \exists v \varphi \Leftrightarrow \exists v (\psi \bullet \varphi)$ for $v \notin \operatorname{FV}(\psi)$.
    \end{itemize}

    \begin{prooftree}
      \AxiomC{$\varphi \Rightarrow \psi$}
      \AxiomC{$\psi \Rightarrow \theta$}
      \BinaryInfC{$\varphi \Rightarrow \theta$}
    \end{prooftree}

    \begin{prooftree}
      \AxiomC{$\varphi \Rightarrow \psi$}
      \UnaryInfC{$\theta \bullet \varphi \Rightarrow \theta \bullet \psi$}
    \end{prooftree}

    \begin{prooftree}
      \AxiomC{$\varphi \Rightarrow \psi$}
      \AxiomC{$v \notin \operatorname{FV}(\psi)$}
      \BinaryInfC{$\exists v \varphi \Rightarrow \psi$}
    \end{prooftree}
  \end{flushleft}
\end{minipage}\hfill
\begin{minipage}{0.5\textwidth}
  \begin{flushright}
    \begin{itemize}
      \item $\varphi[v := w] \Rightarrow \exists v \varphi$,
      \item $\varphi \Rightarrow \bigvee \Phi$ for $\varphi \in \Phi$,
      \item $\bigvee \limits_{\varphi \in \Phi} \varphi \bullet \psi \Leftrightarrow \bigvee \limits_{\varphi \in \Phi} (\varphi \bullet \psi)$,
      \item $\exists v \varphi \bullet \psi \Leftrightarrow \exists v (\varphi \bullet \psi)$ for $v \notin \operatorname{FV}(\psi)$.
    \end{itemize}

    \begin{prooftree}
      \AxiomC{$\varphi \Rightarrow \psi$}
      \UnaryInfC{$\varphi \bullet \theta \Rightarrow \psi \bullet \theta$}
    \end{prooftree}
    

    \begin{prooftree}
      \AxiomC{$\varphi \Rightarrow \psi$}
      \AxiomC{$\varphi \in \Phi$}
      \BinaryInfC{$\bigvee \Phi \Rightarrow \psi$}
    \end{prooftree}

    \begin{prooftree}
      \AxiomC{$\exists v \varphi \Rightarrow \psi$}
      \AxiomC{$v \notin \operatorname{FV}(\psi)$}
      \BinaryInfC{$\varphi \Rightarrow \psi$}
    \end{prooftree}
  \end{flushright}
\end{minipage}
\end{definition}

Let $\mathcal{Q}$ be a quantale and let $D$ be a domain of individuals. 
With every predicate letter $P$ of arity $k < \omega$, we associate its interpretation in $\mathcal{Q}$ which is 
a $k$-ary function $[\![P]\!] : D^k \to \mathcal{Q}$. 
Triples of the form $\mathfrak{Q} = (\mathcal{Q}, D, [\![.]\!])$ are called \emph{quantale models}.
A \emph{variable valuation} is a function $\sigma : \omega \to D$. The value of a geometric formula $\varphi$
in a quantale model $\mathfrak{Q}$ under a valuation $\sigma$ is denoted as $[\![\varphi]\!]^{\mathfrak{Q}}_{\sigma}$
and defined by induction:
\begin{itemize}
  \item $[\![P^k_i(v_{n_1}, \ldots, v_{n_k})]\!]^{\mathfrak{Q}}_{\sigma} = [\![P^k_i]\!]^{\mathfrak{Q}}_{\sigma}(\sigma(n_1), \ldots, \sigma(n_k))$,
  \item $[\![{\bf 1}]\!] = \varepsilon$,
  \item $[\![\varphi \bullet \psi]\!]^{\mathfrak{Q}}_{\sigma} = [\![\varphi]\!]^{\mathfrak{Q}}_{\sigma} \cdot [\![\psi]\!]^{\mathfrak{Q}}_{\sigma}$,
  \item $[\![\bigvee \Phi]\!]^{\mathfrak{Q}}_{\sigma} = \bigvee \limits_{\varphi \in \Phi} [\![\varphi]\!]^{\mathfrak{Q}}_{\sigma}$,
  \item $[\![\exists v_n \varphi]\!]^{\mathfrak{Q}}_{\sigma} = \bigvee \limits_{d \in D} [\![\varphi]\!]^{\mathfrak{Q}}_{\sigma(n \mapsto d)}$.
\end{itemize}
A sequent $\varphi \Rightarrow \psi$ is true in a quantale model $\mathfrak{Q}$ if 
$[\![\varphi]\!]^{\mathfrak{Q}}_{\sigma} \leq [\![\psi]\!]^{\mathfrak{Q}}_{\sigma}$.

\begin{theorem}[Soundness]
  If a sequent $\varphi \vdash \psi$ is provable, then $[\![\varphi]\!]^{\mathfrak{Q}}_{\sigma} \leq [\![\psi]\!]^{\mathfrak{Q}}_{\sigma}$
  in every quantale model.
\end{theorem}

\begin{proof} The proof is standard.
\end{proof}


In particular, when $\mathcal{C}$ is an ordered monoid, that is, for each $a, b \in \mathcal{C}$
the set $\operatorname{Hom}_{\mathcal{C}}(a, b)$ is at most singleton, we instantiate the above construction the following way.

Thus we have:
\begin{prop}
  Every ordered monoid is embeddable to some quantale.
\end{prop}

The proof of the following is a modification of \cite[Theorem 4]{goldblatt2006kripke}.

\begin{theorem}[Completeness]
  If $[\![\varphi]\!]^{\mathfrak{Q}}_{\sigma} \leq [\![\psi]\!]^{\mathfrak{Q}}_{\sigma}$
  in every quantale model $\mathfrak{Q}$, then $\varphi \vdash \psi$ is provable.
\end{theorem}

\begin{proof}
  A \emph{fragment} $\mathcal{F}$ is a set of formulas such that:
  \begin{itemize}
    \item ${\bf 1} \in \mathcal{F}$,
    \item $\varphi, \psi \in \mathcal{F}$ implies $\varphi \bullet \psi \in \mathcal{F}$,
    \item $\varphi \in \mathcal{F}$ implies $\exists v_n \varphi$,
    \item $\varphi \in \mathcal{F}$ implies $\operatorname{Sub}(\varphi) \subseteq \mathcal{F}$,
    \item $\varphi(x) \in \mathcal{F}$ implies $\varphi(x := v_n) \in \mathcal{F}$.
  \end{itemize}

  Any set of formulas $F$ can be extended to a fragment the following way by induction. Construct a sequence of increasing sets:
  \begin{center}
    $F_0 \subseteq F_1 \subseteq \ldots \subseteq F_n \subseteq F_{n + 1} \subseteq \dots$ for $n < \omega$.
  \end{center}
  where
  \begin{itemize}
    \item $F_0 = F \cup At$,
    \item $F_{n + 1} = F_n \cup \{ \varphi \bullet \psi \: | \: \varphi, \psi \in F_n \} \cup \{ \exists v_n \varphi(v_n) \: | \: \varphi(v_n) \in F_n \}$.
  \end{itemize}
  Then we let 
  \begin{center}
  $\mathcal{F} = \bigcup \limits_{n < \omega} F_n$
  \end{center}
  and $\mathcal{F}$ is the smallest fragment extending $F$.

  As usual, we define the following equivalence relation on $\mathcal{F}$
  \begin{center}
    $\varphi \approx \psi$ iff $\varphi \vdash \psi$ and $\psi \vdash \varphi$.
  \end{center}
\end{proof}

\begin{definition}
\end{definition}

TODO: monoidal localisation

\bibliographystyle{alpha}
\bibliography{Text}

\end{document}